
\startcomponent intro-slides
\environment slides-env

\setupcombinations[distance=1cm]
\setupframed[frame=off,height=.9\textheight,width=.5\textwidth,top=\vss,bottom=\vss,align=normal]

\setupTitle
  [ title={ПО сетевых устройств},
   author={Трещановский Павел Александрович, к.т.н.},
     date={\date},
  ]
\placeTitle

\SlideTitle {Пример встроенной системы}
\IncludePicture[horizontal][diagrams/embedded-system-example.jpg]

\SlideTitle {Основные области применения встроенных систем}
Согласно отчету www.marketresearchfuture.com да 2020 год:
\startitemize
\item Automotive
\item Telecommunication
\item Healthcare
\item Industrial
\item Consumer Electronics
\item Military \& Aerospace
\stopitemize

\SlideTitle {Промышленные и транспортные системы}

\startcombination
\framed{\externalfigure[diagrams/industrial-devices2.jpg][width=0.5\textwidth]}{}
\framed
{
\startitemize
\item Двигатели
\item Нефтепроводы, газопроводы
\item Датчики (давления, температуры, влажности и др.)
\item Промышленные контроллеры
\item Мультимедийные системы
\item Позиционирование
\stopitemize
}{}
\stopcombination
\SlideTitle {Телекоммуникационные системы}

\startcombination
\framed{\externalfigure[diagrams/telecom-devices.jpg][width=0.5\textwidth]}{}
\framed
{
\startitemize
\item Коммутаторы
\item Маршрутизаторы
\item Точки доступа
\item Базовые станции
\item Модемы
\stopitemize
}{}
\stopcombination

\SlideTitle {Требования к встроенным системам}
\startitemize
\item Работа в режиме реального времени
\item Высокая производительность (в рамках отведенной задачи)
\item Надежность
\item Низкая мощность и тепловыделение
\item Доверенность
\item Низкая цена
\stopitemize

\SlideTitle {Некоторые факты о встроенных системах}
Christof Ebert, Embedded Software: Facts, Figures and Future:
\startitemize
\item На каждого человека приходится 30 микропроцессоров
\item 98 микропроцессоров находятся во встроенных системах
\item В каждом автомобиле - от 20 до 70 управляющих устройств (100 миллионов
машинных инструкций)
\stopitemize
Прогноз www.marketresearchfuture.com:
\startitemize
\item Объем мирового рынка встроенных систем к 2025 году - 291 миллиард
долларов.
\stopitemize

\SlideTitle {Ключевые тенденции: умный город}
Проблемы:
\startitemize
\item Рост населения
\item Рост урбанизации
\item Исчерпание ресурсов
\item Разрозненность и фрагментарность существующих информационных технологий
\stopitemize

Определение умного города согласно аналитическому агентству Forrester:

A \quote{city} that uses information and communications technologies to make
the critical infrastructure components and services of a city — administration,
education, healthcare, public safety, real estate, transportation, and
utilities — more aware, interactive, and efficient.

\SlideTitle {Составляющие умного города}
\startitemize
\item Коммунальные службы: автоматические дистанционный учет ресурсов, гибкое
распределение ресурсов.
\item Транспорт: предсказание трафика, управление трафиком в зависимости от
текущей нагрузки.
\item Здравоохранение: электронная история болезни, дистанционное наблюдение за
пациентами.
\item Безопасность: видеонаблюдение.
\item Умный дом: мониторинг и управление потреблением тепла, воды, света и др.
\stopitemize

\SlideTitle {Инфраструктура умного города}
\IncludePicture[horizontal][diagrams/smart-city-structure.pdf]

\SlideTitle {Ключевые тенденции: Интернет вещей}
"An open and comprehensive network of intelligent objects that have the
capacity to auto-organize, share information, data and resources, reacting and
acting in face of situations and changes in the environment"

\\

"The Internet [of things] is not only a network of computers, but it has evolved into a
network of devices of all types and sizes, vehicles, smartphones, home appliances,
toys, cameras, medical instruments and industrial systems, all connected, all
communicating and sharing information all the time"

\SlideTitle {Интернет вещей, свойства}
\startitemize
\item Связность: все подключено к глобальной инфраструктуре.
\item Гетерогенность: разнообразие аппаратных платформ и сетевых технологий.
\item Масштаб: на порядок больше устройств, чем в традиционном Интернете.
\item Динамичность: изменение количества и местоположения устройств, изменение
состояния устройств (включение, выключение, засыпание, пробуждение).
\item Защищенность: безопасная передача личных данных через глобальную сеть.
\stopitemize

\SlideTitle {Интернет вещей, архитектура}
\IncludePicture[horizontal][diagrams/iot-architecture.pdf]

\SlideTitle {Ключевые тенденции: искусственный интеллект}
\IncludePicture[horizontal][diagrams/ai-yolo.png]

\SlideTitle {Граничные вычисления (Edge computing)}
\IncludePicture[horizontal][diagrams/edge-computing.png]

\SlideTitle {Преимущества граничных вычислений}
\startitemize
\item Высокая отзывчивость: меньше время передачи данных и больше пропускная
способность.
\item Масштабируемость: снижение нагрузки на облако за счет локальной
предварительной обработки данных.
\item Безопасность: не требуется передача всех данных через глобальную сеть.
\item Надежность: защита от сбоя облака или сети передачи данных.
\stopitemize

\SlideTitle {Ключевые тенденции, замечания}
\startitemize
\item Встроенные системы будут играть ключевую роль.
\item Расширение сетевой инфраструктуры: сеть доходит до каждого физического
объекта.
\item В глобальной сети вещей все устройства будут сетевыми, не только
коммутаторы, маршрутизаторы.
\item В России развитие и внедрение Интернета вещей, искусственного интеллекта
и концепции "Умный дом" будет происходить в рамках национального проекта
"Цифровая экономика". Сквозные цифровые технологии "Новые производственные
технологии", "Технологии беспроводной связи" и др.
\stopitemize

\SlideTitle {Система на кристалле - основа встроенной системы}
\IncludePicture[horizontal][diagrams/typical-soc.pdf]

\SlideTitle {Процессорные ядра ARM и MIPS}
Ядро ARM
\startitemize
\item Архитектура RISC.
\item Существуют 32-битная и 64-битная версия.
\item Старшая линейка ядер - Cortex-A. Типичная частота 1-1.5 ГГц. Количество
ядер на кристалле - 1-16.
\item Производители процессоров: NXP, Qualcomm, Broadcom.
\stopitemize

Ядро MIPS
\startitemize
\item Открытая архитектура.
\item Являются частью российских процессоров "Байкал".
\stopitemize

\SlideTitle {Linux, история}
\IncludePicture[horizontal][diagrams/unix-timeline.png]

\SlideTitle {Linux, факты}
\startcombination
\framed[width=0.3\textwidth]{\externalfigure[diagrams/torvalds.jpg][width=0.3\textwidth]}{}
\framed[width=0.65\textwidth]
{
\startitemize
\item Linux - ядро, а не полная ОС.
\item Первая версия выпущена в 1991 году Линусом Торвальдсом.
\item Поддерживает 25 процессорных архитектур.
\item Масштабируемость: от самых маленьких встроенных систем до суперкомпьютеров.
\item Объем - около 2 миллионов строк кода.
\item Оценочная стоимость разработки - 14 миллиардов долларов.
\stopitemize
}{}
\stopcombination

\SlideTitle {Минимальное программная среда Linux}
\IncludePicture[horizontal][diagrams/minimal-environment.pdf]

\SlideTitle {Расширенная программная среда Linux}
\IncludePicture[horizontal][diagrams/extended-environment.pdf]

\SlideTitle {Пакеты и репозитории}
\IncludePicture[horizontal][diagrams/linux-packages.png]

\SlideTitle {Дистрибутивы Linux}
\startitemize
\item Позволяют создать полноценную операционную систему на основе ядра Linux с
помощью открытого ПО.
\item Предоставляют коллекцию пакетов, оптимизированную под определенную
область применения и процессорную архитектуру.
\item Предоставляют средства создания и установки пакетов.
\item Часто предоставляют средства конфигурирования ОС: настройка системных
служб, сетевые настройки, firewall и др.
\item Известные дистрибутивы: Debian, Ubuntu, Fedora, Gentoo.
\stopitemize

\SlideTitle {Debian, статистика}
\setupTABLE[row][first][align=center]
\setupTABLE[row][each][bottomframe=on]
\bTABLE
\bTABLEhead
\bTR
\bTH \eTH
\bTH lenny (2009) \eTH
\bTH sid (2020) \eTH
\eTR
\eTABLEhead
\bTABLEbody
\bTR
\bTD Место на диске (КБ) \eTD
\bTD 63925088 \eTD
\bTD 332244092 \eTD
\eTR
\bTR
\bTD Количество строк кода \eTD
\bTD 351014627 \eTD
\bTD 1349464674 \eTD
\eTR
\bTR
\bTD Количество пакетов \eTD
\bTD 12517 \eTD
\bTD 31651 \eTD
\eTR
\bTR
\bTD Количество исходных файлов \eTD
\bTD 3713295 \eTD
\bTD 16430196 \eTD
\eTR
\eTABLEbody
\eTABLE

\SlideTitle {Debian, языки программирования}
\IncludePicture[horizontal][diagrams/debian-languages.png]

\SlideTitle {OpenEmbedded - конструктор дистрибутивов}
\IncludePicture[horizontal][diagrams/openembedded.pdf]

\SlideTitle {Системное программирование}
\startitemize
\item Драйверы периферийных устройств
\item Системные утилиты
\item Базы данных
\item Сетевые серверы и клиенты
\item Программы для потоковой обработки данных (в частности, сетевых пакетов)
\item Программы управления датчиками и исполнительными устройствами
\stopitemize

\SlideTitle {AngtelOS в виртуальной машине}
\IncludePicture[horizontal][diagrams/virtualbox.pdf]

\SlideTitle {Редактирование и сборка программ}
\IncludePicture[horizontal][diagrams/editor.pdf]

\SlideTitle {Система сборки make}
\starttyping
цель : зависимости
	shell-команда для создания цели из зависимостей

lab_02: lab_02.o
	gcc -o lab_02 lab_02.o

%.o: %.c
	gcc -O2 -Wall -g -o $@ -c $<

clean:
	rm -f *.o lab_02
\stoptyping

\stopcomponent
