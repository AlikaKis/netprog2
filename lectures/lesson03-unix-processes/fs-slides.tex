
\startcomponent fs-slides
\environment slides-env

\setupTitle
  [ title={ПО сетевых устройств},
   author={Трещановский Павел Александрович, к.т.н.},
     date={\date},
  ]
\placeTitle

\SlideTitle {Виртуальные файлы}
\IncludePicture[horizontal][diagrams/virtual-files.pdf]

\SlideTitle {Файлы блочных и символьных устройств}
\starttyping
# ls -l /dev
brw-rw---- 1 root disk  31,   0 Jan  2 00:54 mtdblock0
crw--w---- 1 root tty    4,  64 Jan  2 00:54 ttyS0
\stoptyping
\startitemize
\item Блочные и символьные файлы не хранят данные. Запросы на чтение и запись
исполняет драйвер соответствующего устройства.
\item Блочные и символьные файлы - не само устройство, а только интерфейс к
нему.
\item Соответствие между файлом и драйвером задается старшими и младшими
номерами устройства ([31, 0] и [4, 64]). Список существующих устройств можно
узнать из файла /proc/devices.
\item Файлы можно создавать вручную:
\starttyping
# mknod myfile c 4 64
\stoptyping
\item Обычно создаются автоматически ядром или приложением udev.
\stopitemize

\SlideTitle {Использование блочных и символьных устройств}
\startitemize
\item Символьные файлы представляют устройства, передающие поток данных.
Пример: терминал /dev/ttyUSB0.
\item Символьные файлы не поддерживают произвольный доступ (флаги {\tt
O_TRUNC}, {\tt O_APPEND}, функция {\tt lseek}).
\item Блочные файлы представляют устройства с произвольным доступом к данным.
Обычно это диски и разделы (/dev/sda, /dev/sda1, /dev/mtdblock3).
\item На ПК диски имеют имена вида sdX, где X (a, b и т.д.) идентифицирует
устройство.
\item Разделы имеют имена вида sdXY, где Y - номер раздела.
\item На встроенной системе разделы ПЗУ имеют имена вида mtdblockX, где X -
номер раздела.
\stopitemize

\SlideTitle {Файловые системы}
\startitemize
\item ПЗУ - физический диск, состоящий из логических разделов. На каждом
разделе - файловая система (FAT, NTFS, ext4, jffs2).
\item Несколько файловых систем объединяются в единое дерево файлов.
\item Каждая файловая система подключается (монтируется) к определенной точке
единого дерева (например, к / или к /home/student).
\item Команда для монтирования ФС:
\starttyping
# mount <устройство_монтируемого_раздела> <точка_монтирования>
\stoptyping
\item Команда mount без аргументов выводит список примонтированных ФС.
\item ФС может быть виртуальной, т.е реализованной в ядре без ПЗУ. Примеры:
proc, sysfs, devtmpfs.
\stopitemize

\SlideTitle {Виртуальные файловые системы (proc и devtmpfs)}
\startitemize
\item Не хранят пользовательские данные. Содержимое может динамически меняться.
Данные не сохраняются при перезагрузке.
\item Первоначальная роль proc - предоставление информации о процессах. {\tt
ps}, {\tt top} получают данные из proc. Обычно монтируется в /proc.
\item Вывод списка файловых дескрипторов процесса <PID>:
\starttyping
# ls -l /proc/<PID>/fd
\stoptyping
\item Сейчас proc также предоставляет общую информацию о системе. Примеры:
список устройств (/proc/devices), список разделов (/proc/partitions).
\item devtmpfs содержит символьные и блочные файлы, создаваемые ядром
автоматически появлении устройств. Обычно монтируется в /dev.
\stopitemize

\SlideTitle {Представление дерева устройств в sysfs}
\startitemize
\item Каждое устройство представлено в виде каталога в sysfs.
\item Устройства имеют двойную классификацию - по шине (PCI, USB) и по
функциональному назначению (последовательный порт, Ethernet-интерфейс).
\item Вывод списка всех Ethernet-интерфейсов:
\starttyping
# ls /sys/class/net
\stoptyping
\item Вывод списка всех устройств на шине I2C:
\starttyping
# ls /sys/bus/i2c/devices
\stoptyping
\item Устройство может иметь читаемые или записываемые файлы (атрибуты). Доступ
реализован в драйвере устройства.
\item Данные в атрибутах генерируются динамически. Кэширование (т.е. {\tt
fopen}, {\tt fread} и т.д.) использовать нельзя!
\stopitemize

\SlideTitle {Сигналы ввода-вывода (GPIO)}
\IncludePicture[horizontal][diagrams/gpio.pdf]

\SlideTitle {Использование GPIO через /sys/class/gpio}
\startitemize
\item Каждый вывод GPIO имеет номер, назначаемый ядром. Номер сам по себе не
имеет физического смысла.
\item Каждому аппаратному блоку выводов GPIO соответствует каталог gpiochipXXX,
где XXX - номер первого вывода в блоке.
\item Атрибуты gpiochip: base - номер первого вывода, ngpio - количество
выводов в блоке.
\item Перед использованием вывод GPIO (например, номер 472) должен быть
экспортирован:
\starttyping
# cd /sys/class/gpio
# echo 472 > export
# ls
export       gpio472      gpiochip472  gpiochip480  unexport
\stoptyping
\item Чтение текущего состояния и установка высокого уровня на выводе 472:
\starttyping
# cat gpio472/value
# echo 1 > gpio472/value
\stoptyping
\stopitemize

\SlideTitle {Время в Linux}
\startitemize
\item Время отсчитывает аппаратный таймер. Этим таймером управляет ядро Linux.
\item Приложение может задержать свое исполнение системным вызовом {\tt usleep}
\starttyping
int usleep(useconds_t usec);
\stoptyping
\item Выход из функции {\tt usleep} означает завершение запрошенной паузы.
\item {\tt usleep} - пример блокирующего вызова, т.е вызова, который
приостанавливает (блокирует) приложение до получения результата.
\item По умолчанию вызовы {\tt read} и {\tt write} являются блокирующими, т.е.
если нет данных для чтения, {\tt read} будет ждать их появления.
\stopitemize

\SlideTitle {Упражнения}
\startitemize
\item Написать программу для чтения и вывода в терминал текущего состояния
вывода GPIO. Номер GPIO задается единственным позиционным аргументом.
\item Написать программу для установки значения вывода GPIO. Номер GPIO
задается первым позиционным аргументом. Новое значение - вторым.
\item Написать программу, которая 4 раза меняет значение GPIO на
противоположное с интервалом в 1 секунду. Номер GPIO задается единственным
позиционным аргументом.
\stopitemize

\stopcomponent
