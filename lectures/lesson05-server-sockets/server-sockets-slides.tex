
\startcomponent server-sockets-slides
\environment slides-env

\setupTitle
  [ title={ПО сетевых устройств},
   author={Трещановский Павел Александрович, к.т.н.},
     date={\date},
  ]
\placeTitle

\SlideTitle {Сервер на основе пакетного сокета}
\IncludePicture[horizontal][diagrams/dgram-server.pdf]

\SlideTitle {Привязка сокета к адресу}
\startitemize
\item Пример:
\starttyping
	struct sockaddr_un addr;
	addr.sun_family = AF_UNIX;
	snprintf(addr.sun_path, sizeof(addr.sun_path), "%s", "/tmp/sock_dgram_server");

	bind(sock_fd, (struct sockaddr *)&addr, sizeof(addr));
\stoptyping
\item Для каждого семейства адресов своя структура, описывающая адрес ({\tt
sockaddr_un}, {\tt sockaddr_in}). При передаче в качестве аргумента {\tt bind},
адрес приводится к абстрактному типу {\tt sockaddr}.
\item {\tt bind} для Unix-сокета создает файл особого типа с указанным именем.
\item Файл не хранит данные!
\item Файл остается после закрытия сокета. Перед повторным вызовом {\tt bind}
его необходимо удалить.
\stopitemize

\SlideTitle {Обработка запросов на сервере}
\startitemize
\item У клиента обычно нет публичного адреса.
\item Сервер отправляет ответ на тот адрес, с которого пришел запрос.
\item Пример:
\starttyping
	struct sockaddr_un addr;
	socklen_t addrlen = sizeof(addr);

	recvfrom(sock_fd, rx_buf, rx_len, (struct sockaddr *)&addr, &addrlen);

	sendto(sock_fd, tx_buf, tx_len, (struct sockaddr *)&addr, addrlen);
\stoptyping
\item {\tt addrlen} должен быть инициализирован до вызова {\tt recvfrom}. Он
указывает размер буфера для хранения адреса. После вызова он указывает реальный
размер адреса.
\stopitemize

\SlideTitle {Отличия потокового сервера от пакетного}
\startitemize
\item Потоковый сокет требует установки соединения. Что дает соединение:
\startitemize
\item Сохранение порядка передачи данных за счет нумерации байтов потока.
\item Надежная доставка за счет повторной отправки неподтвержденных сегментов.
\item Детектирование разрыва соединения.
\stopitemize
\item Для AF_INET соединение устанавливается с помощью протоколу TCP, для
AF_UNIX - средствами ядра Linux.
\item Соединение устанавливается между 2 приложениями.
\item Проблема: Каждый сокет может принадлежать 1 соединению, но серверу
необходимо обслуживать несколько клиентов (несколько соединений).
\stopitemize

\SlideTitle {Слушающий сокет}
\starttyping
       int listen(int sockfd, int backlog);
\stoptyping
\startitemize
\item Слушающий сокет отвечает только за прием соединений от клиентов.
\item Функция {\tt listen} переводит сокет, созданный с помощью {\tt socket}, в
слушающий режим.
\item Слушающий сокет необходимо привязывать к публичному адресу.
\item Аргумент {\tt backlog} задает максимальное количество соединений, ждущих
приема. При превышения этого количества запрос на соединение отклоняется.
\item Прием соединения и извлечение его из очереди {\tt backlog} осуществляется
функцией {\tt accept}.
\stopitemize

\SlideTitle {Передающий сокет}
\starttyping
       int accept(int sockfd, struct sockaddr *addr, socklen_t *addrlen);
\stoptyping
\startitemize
\item Функция {\tt accept} возвращает дескриптор передающего сокета.
\item Передача данных по каждому из установленных соединений производится через
передающие сокеты. 1 передающий сокет = 1 соединение.
\item В приложении может существовать несколько передающих сокетов, созданных
из одного слушающего сокета.
\item {\tt accept} - блокирующая функция, которая возвращается только после
приема нового соединения.
\item Передающий сокет уже привязан (вызывать {\tt bind} не нужно).
\item Передача и прием производятся функциями {\tt send} и {\tt recv}.
\item Когда клиент разрывает соединение, {\tt recv} на передающем сокете
сервера возвращает 0.
\stopitemize

\SlideTitle {Сервер на основе потокового сокета}
\IncludePicture[horizontal][diagrams/stream-server.pdf]

\SlideTitle {Алгоритм сервера с одним передающим потоком}
\IncludePicture[horizontal][diagrams/stream-server-algorithm.pdf]

\SlideTitle {Упражнения}
\startitemize
\item Разработать сервер на основе пакетного Unix-сокета, который принимает
запросы на адресе /tmp/sock_dgram_server, и для каждого запроса выводит
сообщение в терминал. Анализировать запросы и отправлять ответы не требуется.
\item Разработать сервер на основе потокового Unix-сокета, который принимает
соединения на адресе /tmp/sock_stream_server, и для каждого соединения выводит
сообщение в терминал. Соединение следует сразу же закрывать.
\stopitemize

\stopcomponent
