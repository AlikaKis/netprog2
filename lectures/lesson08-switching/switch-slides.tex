
\startcomponent switch-slides
\environment slides-env

\setupTitle
  [ title={ПО сетевых устройств},
   author={Трещановский Павел Александрович, к.т.н.},
     date={\date},
  ]
\placeTitle

\SlideTitle {Сетевой, канальный и физический уровни}
\IncludePicture[horizontal][diagrams/global-network.pdf]

\SlideTitle {Локальные сети}
\startitemize
\item В локальных сетях поддерживается широковещательная (т.е. всем узлам сети)
рассылка.
\item Зачем нужна широковещательная рассылка? Для разрешения IP-адресов в
MAC-адреса. Т.е. спрашиваем всех, где находится IP-адрес X.X.X.X, обладатель
этого адреса отвечает и сообщает свой MAC-адрес.
\item Если сеть с общей средой передачи, то все кадры широковещательные
(например, WiFi).
\item Почти все локальные сети - Ethernet или родственные (WiFi).
\item Канальный уровень применительно к Ethernet также называется MAC - Media
Access Control (управление доступом к среде передачи).
\item Физичиские уровни (PHY) для Ethernet: 100Base-TX, 1000Base-T (медь),
100Base-FX, 1000Base-X (оптика).
\stopitemize

\SlideTitle {Структура Ethernet-кадра}
\externalfigure[diagrams/ethernet-frame.pdf][width=\textwidth]
\startitemize
\item Если значение Ethertype/Length меньше 0x800, то поле содержит длину
кадра. В противном случае - тип кадра (IP, ARP и т.д.).
\item Первые 3 байта MAC-адреса - OUI (Organizationally Unique Identifier), код
производителя.
\item CRC32 обычно устанавливается и проверяется на аппаратном уровне.
\item Размер кадра - от 64 до 1522 байтов.
\stopitemize

\SlideTitle {Передача Ethernet-кадра}
\IncludePicture[horizontal][diagrams/frame-transmission.pdf]

\SlideTitle {Сеть на основе повторителя}
\IncludePicture[horizontal][diagrams/repeater.pdf]

\SlideTitle {Коллизия в полудуплексных сетях}
\IncludePicture[horizontal][diagrams/collision.pdf]

\SlideTitle {Сеть на основе коммутатора}
\IncludePicture[horizontal][diagrams/switch.pdf]

\SlideTitle {Буферизация кадров}
\IncludePicture[horizontal][diagrams/switch-buffer.pdf]

\SlideTitle {Изучение MAC-адресов}
\IncludePicture[horizontal][diagrams/learning.pdf]

\SlideTitle {Устаревание MAC-адресов}
\IncludePicture[horizontal][diagrams/address-aging.pdf]

\SlideTitle {Коммутатор Корунд}
\IncludePicture[horizontal][diagrams/managed-switch.pdf]

\SlideTitle {Сокеты семейства AF_PACKET}
\starttyping
	#include <linux/if_packet.h>
	#include <net/ethernet.h>

	int sock;
	struct sockaddr_ll bind_addr, rx_addr;
	char pkt_buf[2048];
	struct ether_header *eth;
	struct ether_addr *da;

	sock = socket(AF_PACKET, SOCK_RAW, 0);

	memset(&bind_addr, 0, sizeof(bind_addr));
	bind_addr.sll_family   = AF_PACKET;
	bind_addr.sll_protocol = htons(ETH_P_ALL);
	bind_addr.sll_ifindex  = ifindex; /* Индекс можно найти в выводе ip addr */
	bind(sock, (struct sockaddr*)&bind_addr, sizeof(bind_addr));

	recvfrom(sock, pkt_buf, sizeof(pkt_buf), 0, &rx_addr, sizeof(rx_addr));
	if (rx_addr.sll_pkttype == PACKET_OTHERHOST) {
		eth = pkt_buf;
		da = (struct ether_addr *)eth->ether_dhost;
		printf("Destination address: %s\n", ether_ntoa(a));
	}
\stoptyping

\SlideTitle {Виртуализация сетевого стека}
\startitemize
\item Если в Linux настроено два интерфейса eth0 и eth1 192.168.0.8 и
192.168.0.9, то команда ping 192.168.0.9 {\tt не} будет передавать запрос через
eth0.
\item Почему - общая таблица маршрутизации. Linux знает, что оба адреса локальные.
\item Нужна виртуализация. В Linux'е виртуализация осуществляется с помощью
контейнеров (aka LXC-контейнеры).
\item Контейнер виртуализирует среду исполнения процесса, а не всей ОС.
\item Контейнер состоит из пространств имен (namespace).
\item Есть пространства имен для процессов, пользователей, сетевого стека.
\item Пространства имен можно использовать по отдельности, без полной
виртуализации.
\stopitemize

\SlideTitle {Запуск приложений в определенном пространстве имен}
Вывод списка интерфейсов в сетевом пространстве имен /etc/ns/ns1:
\starttyping
# nsenter --net=/etc/ns/ns1 ip addr
1: lo: <LOOPBACK> mtu 65536 qdisc noop state DOWN qlen 1000
    link/loopback 00:00:00:00:00:00 brd 00:00:00:00:00:00
4: ge2@if2: <NO-CARRIER,BROADCAST,MULTICAST,UP,M-DOWN> ...
    link/ether 68:eb:c5:00:01:02 brd ff:ff:ff:ff:ff:ff
    inet 192.168.0.10/24 scope global ge2
       valid_lft forever preferred_lft forever
\stoptyping
Замечание. Имена /etc/ns/nsX специфичны для Корунда. На других системах
именование пространств имен будет отличаться.

\SlideTitle {Схема виртуальной локальной сети}
\IncludePicture[horizontal][diagrams/virtual-network.pdf]

\SlideTitle {Замечания}
\startitemize
\item Тестирование осуществляется с помощью утилиты ping.
\item Аргументом ping является IP-адрес. Чтобы узнать MAC-адрес, Linux
отправляет широковещательный ARP-запрос. Обладатель указанного адреса
возвращает ARP-ответ. Только после этого отправляется ICMP-запрос.
\item Что должно быть обязательно проверено.
\startitemize
\item Если адрес не изучен, кадр отправляется на все порты.
\item Если изучен, то только на один.
\item Если поменять порт подключения виртуального компьютера, к которому
периодически отправляются ICMP-запросы, то должна произойти временная потеря
связи и последующим автоматическим восстановлением.
\stopitemize
\stopitemize

\stopcomponent
