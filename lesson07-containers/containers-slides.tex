
\startcomponent containers-slides
\environment slides-env

\setupTitle
  [ title={ПО сетевых устройств},
   author={Трещановский Павел Александрович, к.т.н.},
     date={\date},
  ]
\placeTitle

\SlideTitle {Структура сетевого приложения}
\IncludePicture[horizontal][diagrams/app-structure.pdf]

\SlideTitle {Контейнеры}
Что нужно?
\startitemize
\item Динамическое добавление и удаление элементов.
\item Извлечение элемента по номеру или ключу.
\item Перебор всех элементов контейнера.
\stopitemize

Виды контейнеров.
\startitemize
\item Списки - последовательность элементов, связанных указателями.
Обеспечивает быстрое добавление и поиск по порядковому номеру.
\item Словари (ассоциативные массивы) - множество пар ключ-значение.
Обеспечивает быстрый поиск по ключу.
\stopitemize

\SlideTitle {Простой список}
\IncludePicture[horizontal][diagrams/simple-list.pdf]

\SlideTitle {Добавление и удаление элементов списка}

\startcolumns[n=2]
\starttyping
/* Добавление элемента new в начало */
struct list_head head;
struct list_head new;
...
new.prev = &head;
new.next = head.next
head.next->prev = &new;
head.next = &new;
/* То же самое: list_add(&new, &head); */
\stoptyping
\column
\starttyping
/* Удаление элемента new из списка */
struct list_head new;
...
new.prev->next = new.next;
new.next->prev = new.prev;
/* То же самое: list_del(&new); */
\stoptyping
\stopcolumns
\externalfigure[diagrams/addition-deletion.pdf][width=\textwidth]

\SlideTitle {Перебор элементов простого списка}
\starttyping
struct list_head {
	struct list_head *next;
	struct list_head *prev;

	struct user_object user_data;
} head, *ptr;

for (ptr = head.next;
     ptr->next != &head; \
     ptr = ptr->next) {

     struct user_object *uobj = &ptr->user_data;
     /* Операции с uobj */
}
\stoptyping
 
\SlideTitle {Универсальный список}
\IncludePicture[horizontal][diagrams/generic-list.pdf]

\SlideTitle {Получение указателя на родительский элемент}
\externalfigure[diagrams/container-of.pdf][width=\textwidth]
 
\starttyping
/* Смещение между адресом родительского объекта типа type и адресом его поля
member. */
#define offsetof(type, member)	((size_t)&((type *)0)->member)

/* Получение указателя на родительский объект типа type из указателя ptr на его
поле member. */
#define container_of(ptr, type, member) ({		\
	void *__mptr = (void *)(ptr);			\
	((type *)(__mptr - offsetof(type, member))); })
\stoptyping

\SlideTitle {Перебор элементов универсального списка}
\starttyping
struct list_head head;
struct user_object {
	void *data;
	struct list_head entry;
} *uobj;

for (uobj = container_of(head.next, typeof(*uobj), entry);
     &uobj->entry != &head; \
     uobj = container_of(uobj->entry.next, typeof(*uobj), entry)) {
     /* Операции с uobj */
}

/* Тот же самый перебор. */
list_for_each_entry(uobj, &head, entry) {
     /* Операции с uobj */
}
\stoptyping
 
\SlideTitle {Использование {\tt struct list_head}}
\starttyping
struct list_head list;
struct user_object {
	int data;
	struct list_head entry;
} *uobj1, *uobj2;

INIT_LIST_HEAD(&list);

uobj1 = malloc(sizeof(*uobj1));
uobj1->data = 7857;

list_add(&uobj1->entry, &list);

uobj2 = list_first_entry(&list, struct user_object, entry);

list_for_each_entry(uobj2, &list, entry) {
	printf("%d\n", uobj2->data);
}
\stoptyping
\SlideTitle {Поиск элемента по ключу}
\IncludePicture[horizontal][diagrams/key-search.pdf]
 
\SlideTitle {Дерево поиска}
\IncludePicture[horizontal][diagrams/search-tree.pdf]
 
\SlideTitle {АВЛ-дерево}
\IncludePicture[horizontal][diagrams/avl-tree.pdf]

\SlideTitle {Использование {\tt struct avl_tree} и {\tt struct avl_node}}
\starttyping
struct avl_tree dict;
struct user_object {
	int data;
	char *key;
	struct avl_node anode;
} *uobj1, *uobj2;

avl_init(&dict, avl_strcmp, 0, NULL);

uobj1 = malloc(sizeof(*uobj1));
uobj1->data = 7857;
uobj1->key = strdup("some_key");
uobj1->anode.key = uobj1->key;

avl_insert(&dict, &uobj1->anode);

uobj2 = avl_find_element(&dict, "another_key", uobj2, anode);

avl_for_each_element(&dict, uobj2, anode) {
	printf("%s\n", uobj2->key);
}
\stoptyping
 
\SlideTitle {Удаление элементов во время перебора}
\starttyping
struct list_head list;
struct user_object {
	int data;
	struct list_head entry;
} *uobj, *next_uobj;

/* Неправильно! */
list_for_each_entry(uobj, &list, entry) {
	list_del(&uobj->entry);
	free(uobj);
	/* Дальше произойдет Segmentation fault, потому что указатель на
	следующий элемент должен получаться из текущего эелемента, который уже
	удален. */
}

/* Правильно */
list_for_each_entry_safe(uobj, next_uobj, &list, entry) {
	list_del(&uobj->entry);
	free(uobj);
}
\stoptyping

\SlideTitle {Упражнения}
\startitemize
\item Разработать приложение, создающее список на основе {\tt struct list_head}
и добавляющее в него строковые элементы "aaa", "bbb", ..., "zzz". Приложение
должно вывести в терминал элемент списка под номером n, где n - значение
первого аргумента командной строки. После вывода приложение должно удалить все
элементы списка.
\item Разработать приложение, создающее словарь на основе {\tt struct alv_tree}
и добавляющее в него строковые элементы "aaa", "bbb", ..., "zzz" с ключами 'a',
'b', ..., 'z' . Приложение должно вывести в терминал элемент словаря с ключом
k, где k - значение первого аргумента командной строки. После вывода приложение
должно удалить все элементы словаря.
\stopitemize

\stopcomponent
