
\startcomponent kernel-drivers-slides
\environment slides-env

\setupTitle
  [ title={ПО сетевых устройств},
   author={Трещановский Павел Александрович, к.т.н.},
     date={\date},
  ]
\placeTitle

\SlideTitle {Ядро Linux}
\startitemize
\item Ядро имеет прямой доступ к аппаратной периферии.
\item Ядро реагирует на внешние события (прерывания).
\item Загрузчик записывает ядро в ОЗУ во время запуска системы. Далее ядро
всегда целиком находится в ОЗУ.
\item  Основная работа ядра выполняется по запросам от приложений (системным
вызовам). Похоже на библиотеку.
\item Вспомогательная работа выполняется в процессах ядра (в выводе {\tt ps}
имена в квадратных скобках).
\item Код ядра полностью автономен. Не используется даже стандартная библиотека
C.
\item После ошибки ядро не восстанавливается. Требуется перезагрузка.
\stopitemize

\SlideTitle {Модули ядра}
Загрузка и выгрузка из ядра - командами {\tt insmod} и {\tt rmmod}. Просмотр -
{\tt lsmod}.

\starttyping
#include <linux/module.h>
#include <linux/kernel.h>
#include <linux/slab.h>

static char *buf;
static int example_init(void)
{
	printk("Hello world\n");
	buf = kzalloc(100, GFP_KERNEL);
	if (!buf)
		return -ENOMEM;
	return 0;
}
static void example_exit(void)
{
	free(buf);
}

module_init(example_init);
module_exit(example_exit);
\stoptyping

\SlideTitle {Аппаратные шины}
\IncludePicture[horizontal][diagrams/hardware-buses.pdf]

\SlideTitle {Структура физического адресного пространства}
\IncludePicture[horizontal][diagrams/address-space.pdf]

\SlideTitle {Обращение к регистрам в C}
\starttyping
#define DEVICE_BASE_ADDR    0x06380000
#define DEVICE_CONFIG_REG   0x0
#define DEVICE_STATUS_REG   0x1
#define DEVICE_INTR_REG     0x2

void read_register(void)
{
	void *base_addr = (void *)DEVICE_BASE_ADDR;
	unsigned int config, status;

	config = *(unsigned int *)(base_addr + DEVICE_CONFIG_REG);
	status = *(unsigned int *)(base_addr + DEVICE_STATUS_REG);
}
\stoptyping

\SlideTitle {Структура физического адресного пространства}
\IncludePicture[horizontal][diagrams/virtual-addresses.pdf]

\SlideTitle {Регистрация и связывание драйвера}
\starttyping
#include <linux/platform_device.h>
#include <linux/of_device.h>

static const struct of_device_id a100_example_of_match[] = {
	{ .compatible = "ctlm,a100-example", }, {}
};
/* platform - название внутренней процессорной шины. */
static struct platform_driver a100_example_drv = {
	.driver = {
		.owner = THIS_MODULE,
		.name = "a100_example",
		.of_match_table = a100_example_of_match, /* Признак для связывания */
	},
};
static int a100_example_init(void)
{
	return platform_driver_register(&a100_example_drv);
}
static void a100_example_exit(void)
{
	platform_driver_unregister(&a100_example_drv);
}
\stoptyping

\SlideTitle {Дерево устройств (Device Tree)}
\startitemize
\item  Структура данных, описывающая недетектируемые устройства (SPI, I2C,
устройства на внутренней шине) и отношения между ними.
\item Хранится в ПЗУ устройства. Во время запуска системы загрузчик передает
дерево устройств ядру.
\item of_ - OpenFirmware (стандарт, в котором первоначальном был описан формат
Device Tree).
\item Пример:
\starttyping
soc {
	/* ... */
	a100-example {
		compatible = "ctlm,a100-example";
		reg = <0x063C0000 0x010000>; /* Базовый адрес и размер
		                                банка регистров. */
		interrupts = <4>, <5>;
	};
};
\stoptyping
\stopitemize

\SlideTitle {Инициализация устройства}
\starttyping
#include <linux/io.h>

static int a100_example_probe(struct platform_device *pdev)
{
	struct resource *res;
	void __iomem *base;

	/* Получение базового адреса банка регистров. */
	res = platform_get_resource(pdev, IORESOURCE_MEM, 0);

	/* Отображение в виртуальное адресное пространство. */
	base = ioremap(res->start, resource_size(res));
	return 0;
}
static int a100_example_remove(struct platform_device *pdev) {/* ... */}

static struct platform_driver a100_example_drv = {
	.probe = a100_example_probe,
	.remove = a100_example_remove,
};
\stoptyping

\SlideTitle {Привязка объектов к {\tt struct platform_device}}
\starttyping
static int a100_example_probe(struct platform_device *pdev)
{
	struct a100_example_data *data;
	struct resource *res;
	void __iomem *base;

	data = kzalloc(sizeof(struct a100_example_data), GFP_KERNEL);
	/* ... */
	data->base = ioremap(res->start, resource_size(res));

	platform_set_drvdata(pdev, data);
	return 0;
}
static int a100_example_remove(struct platform_device *pdev)
{
	struct a100_example_data *data = platform_get_drvdata(pdev);

	iounmap(data->base);
	free(data);
	return 1;
}
\stoptyping

\SlideTitle {Чтение и запись регистров}
\startitemize
\item Чтение регистра с базовым адресом {\tt base} и смещением 5:
\starttyping
#include <linux/io.h>

void __iomem *base;
u32 reg;
reg = readl(base + 5);
\stoptyping
\item Запись значения {\tt val} в регистр с базовым адресом {\tt base} и смещением 5:
\starttyping
u32 val;
writel(val, base + 5);
\stoptyping
\item Установка бита 6 и снятие бита 7:
\starttyping
u32 reg;
reg = readl(base + 5);
reg |= (1 << 6);
reg &= ~(1 << 7);
writel(reg, base + 5);
\stoptyping
\stopitemize

\SlideTitle {Информация об устройстве в sysfs}
\startitemize
\item Список драйверов для шины platform:
\starttyping
# ls /sys/bus/platform/drivers
\stoptyping
\item Список устройств на шины platform:
\starttyping
# ls /sys/bus/platform/devices
\stoptyping
\item Каждому устройству соответствует каталог в sysfs.
\item Файлы в каталоге - атрибуты.
\item Все файлы виртуальные. Данные не хранятся на жестком диске.
\stopitemize

\SlideTitle {Определение новых атрибутов}
\starttyping
ssize_t example_show(struct device *dev, struct device_attribute *attr, char *buf)
{
	struct a100_example_data *data = platform_get_drvdata(to_platform_device(dev));

	/* То, что записывается в buf, передается приложению через системный
	вызов read. */
	return sprintf(buf, "%s", "Hello world");
}

ssize_t example_store(struct device *dev, struct device_attribute *attr,
		      const char *buf, size_t len)
{
	struct a100_example_data *data = platform_get_drvdata(to_platform_device(dev));
	char *endp;

	/* То, что приложение передает в системный вызов write, оказывается в buf. */
	data->value = kstrtoul(buf, &endp, 0);

	return len;
}

static DEVICE_ATTR_RW(example); /* Имя атрибута совпадает с префиксом имен функций. */
\stoptyping

\SlideTitle {Регистрация атрибутов}
\starttyping
static DEVICE_ATTR_RW(example0);
static DEVICE_ATTR_RO(example1);

static struct device_attribute *example_attrs[] = {
	&dev_attr_example0,
	&dev_attr_example1,
	NULL,
};
static struct attribute_group example_group = {
	.name = "example group",
	.attrs = example_attrs,
}

static int a100_example_probe(struct platform_device *pdev)
{
	/* ... */

	sysfs_create_group(&pdev->dev.kobj, example_group);
}
\stoptyping

\SlideTitle {Обработка событий}
\IncludePicture[horizontal][diagrams/interrupts.pdf]

\SlideTitle {Регистрация обработчика прерывания}
\starttyping
#include <linux/interrupt.h>

static irqreturn_t a100_example_isr(int irq, void *id)
{
	struct a100_example_data *data = id;
	return IRQ_HANDLED;
}
static int a100_example_probe(struct platform_device *pdev)
{
	struct a100_example_data *data;
	int irq;

	/* Получение внутреннего номера для 0-го прерывания. */
	irq = platform_get_irq(pdev, 0);

	/* Регистрация обработчика для прерывания irq. Обработчику при вызове
	   будет передан указатель data. */
	request_irq(irq, a100_example_isr, 0, "a100-example-isr", data);
	return 0;
}
\stoptyping

\SlideTitle {Мультиплексирование прерываний}
\IncludePicture[horizontal][diagrams/irq-mux.pdf]

\SlideTitle {Аппаратный таймер}
\IncludePicture[horizontal][diagrams/timer.pdf]

\stopcomponent
